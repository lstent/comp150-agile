% Please do not change the document class
\documentclass{scrartcl}

% Please do not change these packages
\usepackage[hidelinks]{hyperref}
\usepackage[none]{hyphenat}
\usepackage{setspace}
\doublespace

% You may add additional packages here
\usepackage{amsmath}

% Please include a clear, concise, and descriptive title
\title{The allocation of tasks using Agile methodologies (e.g. Scrum) in an amateur team is uneven and/ or biased}

% Please do not change the subtitle
\subtitle{COMP150 - Agile Development Practice}

% Please put your student number in the author field
\author{1506919}

\begin{document}

\maketitle

\abstract{Agile methodologies (e.g. Scrum) in an amateur team is uneven and/or biased More specifically how in a classroom setting the allocation of tasks becomes very uneven and biased, as there is no financial benefit or maybe there is no interest in the task, which has a negative effect on the team collaboration in Scrum, this paper adressing the topic by looking at case studies based in a classroom setting in which the students are using an agile methodology for the first time, looking in depth at the collaboration between the student groups, comparing how professionals using agile methodologies compare to amateurs. Hopefully finding a better method to teach students how to collaborate in a more professional way.}

\section{Introduction}

This paper will discuss the implementation of agile methodologies \cite{beck2001manifesto} (focusing on scrum) in non-professional teams, particularly within the classroom setting, and how the scrum can quickly become biased \cite{kropp2014teaching}. By which I mean one group member may be given the majority of the workload or the more complex of the scrum tasks, while another member is given an easier task to complete in the same time period, the chances of this occuring is greatly increased if there is a member with an introverted personality, compared to Agile being used properly in a professional games industy. After outlining these problems possible solutions in the form of practice collaboration tasks, teaching the corelation of daily stand up meetings, positivity and team productivity \cite{kropp2016teaching}.

\section{Agile in the professional game studio}

The size of teams needed to produce a triple-A (AAA) game increases every year creating uncertainty as to which management technique is best for them \cite{mateos2008adopting} \cite{mcguire2006paper}. Use of Agile in managing spoftware projects is becoming increasingly popular\cite{one2012state} an example of this is in game studios, in which everyone contributes different parts to make a successful game. The waterfall method of managing software which is designed by keeping each team within the same occupation, that is the artists in one group, programmers in another, then the game gets tested. All the problems are logged then the waterfall is repeated, the main problems that occured is the notes from one team to the other could be misunderstood \cite{snapp2008accidental}. Agile differs from this by instead of seperating the different roles, creating teams with workers from each of the different diciplines, which in effect create small game studios responsible for their own part of the full working game, new tasks are set each month for every small team instead of trying to tackle every task at once. The project leader is responsible for setting tasks and making sure everything is ready on time for the release date \cite{Crowley2016GDC}. Agile enables fast user feedback during the development process, used with scrum which modifies the priority list, decreasing the time spent on game development \cite{mateos2008adopting}.

\section{Agile in the classroom}



\bibliographystyle{ieeetran}
\bibliography{references}

\end{document}
