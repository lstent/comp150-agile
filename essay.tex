% Please do not change the document class
\documentclass{scrartcl}

% Please do not change these packages
\usepackage[hidelinks]{hyperref}
\usepackage[none]{hyphenat}
\usepackage{setspace}
\doublespace

% You may add additional packages here
\usepackage{amsmath}

% Please include a clear, concise, and descriptive title
\title{The allocation of tasks using Agile methodologies (e.g. Scrum) in an amateur team is uneven and/ or biased}

% Please do not change the subtitle
\subtitle{COMP150 - Agile Development Practice}

% Please put your student number in the author field
\author{1506919}

\begin{document}

\maketitle

\abstract{Agile methodologies (e.g. Scrum) in an amateur team is uneven and/or biased More specifically how in a classroom setting the allocation of tasks becomes very uneven and biased, as there is no financial benefit or maybe there is no interest in the task, which has a negative effect on the team collaboration in Scrum, I tend to address this question by looking at case studies based in a classroom setting in which the students are using an agile methodology for the first time, looking in depth at the collaboration between the student groups, comparing how professionals using agile methodologies compare to amateurs. Hopefully finding a better method to teach students how to collaborate in a more professional way.}

\section{Introduction}

In this paper I will discuss the implementation of agile methodologies (focusing on scrum) in non-professional teams, particularly within the classroom setting, and how the scrum can quickly become biased. By which I mean one group member may be given the majority of the workload or the more complex of the scrum tasks, while another  member is given an easier task to complete in the same time period, the chances of this occuring is greatly increased if there is a member with an introverted personality. After outlining these problems I will provide a solution in the form of practice collaboration tasks, teaching the corelation of daily stand up meetings and team positivity .

\bibliographystyle{ieeetran}
\bibliography{references}

\end{document}
